%%Macro package for typeseting set partitions
%%%%%%%%%%%%%%%%%%%%%%%%%%%%%%%%%%%%%%%%%%%%%

%First, also as an example, we define shortcuts for some short partitions using word representation.

%%%Partitions with lower points only

%%k=1

\def\La{\LPartition{0.6:1}{}}

%%k=2

\def\Laa{\LPartition{}{0.6:1,2}}
\def\Lab{\LPartition{0.6:1,2}{}}

%k=3

\def\Laaa{\LPartition{}{0.6:1,2,3}}
\def\Laab{\LPartition{0.6:3}{0.6:1,2}}
\def\Laba{\LPartition{0.4:2}{0.8:1,3}}
\def\Labb{\LPartition{0.6:1}{0.6:2,3}}
\def\Labc{\LPartition{0.6:1,2,3}{}}

%k=4


\def\Laaaa{\LPartition{}{0.6:1,2,3,4}}
\def\Laaab{\LPartition{0.6:4}{0.6:1,2,3}}
\def\Laaba{\LPartition{0.4:3}{0.8:1,2,4}}
\def\Labaa{\LPartition{0.4:2}{0.8:1,3,4}}
\def\Labbb{\LPartition{0.6:1}{0.6:2,3,4}}
\def\Laabb{\LPartition{}{0.6:1,2;0.6:3,4}}
\def\Labba{\LPartition{}{0.8:1,4;0.4:2,3}}
\def\Labab{\LPartition{}{0.8:2,4;0.4:1,3}}
\def\Laabc{\LPartition{0.6:3,4}{0.6:1,2}}
\def\Labac{\LPartition{0.4:2,4}{0.8:1,3}}
\def\Labca{\LPartition{0.4:2,3}{0.8:1,4}}
\def\Labbc{\LPartition{0.6:1,4}{0.6:2,3}}
\def\Labcb{\LPartition{0.4:1,3}{0.8:2,4}}
\def\Labcc{\LPartition{0.6:1,2}{0.6:3,4}}
\def\Labcd{\LPartition{0.6:1,2,3,4}{}}

%%%Partitions with upper points only

%%k=1

\def\Ua{\UPartition{0.4:1}{}}

%%k=2

\def\Uaa{\UPartition{}{0.4:1,2}}
\def\Uab{\UPartition{0.4:1,2}{}}

%k=3

\def\Uaaa{\UPartition{}{0.5:1,2,3}}
\def\Uaab{\UPartition{0.4:3}{0.4:1,2}}
\def\Uaba{\UPartition{0.6:2}{0.2:1,3}}
\def\Uabb{\UPartition{0.4:1}{0.4:2,3}}
\def\Uabc{\UPartition{0.4:1,2,3}{}}

%k=4

\def\Uaaaa{\UPartition{}{0.4:1,2,3,4}}
\def\Uaaab{\UPartition{0.4:4}{0.4:1,2,3}}
\def\Uaaba{\UPartition{0.6:3}{0.2:1,2,4}}
\def\Uabaa{\UPartition{0.6:2}{0.2:1,3,4}}
\def\Uabbb{\UPartition{0.4:1}{0.4:2,3,4}}
\def\Uaabb{\UPartition{}{0.4:1,2;0.4:3,4}}
\def\Uabba{\UPartition{}{0.2:1,4;0.6:2,3}}
\def\Uabab{\UPartition{}{0.2:2,4;0.6:1,3}}
\def\Uaabc{\UPartition{0.4:3,4}{0.4:1,2}}
\def\Uabac{\UPartition{0.6:2,4}{0.2:1,3}}
\def\Uabca{\UPartition{0.6:2,3}{0.2:1,4}}
\def\Uabbc{\UPartition{0.4:1,4}{0.4:2,3}}
\def\Uabcb{\UPartition{0.6:1,3}{0.2:2,4}}
\def\Uabcc{\UPartition{0.4:1,2}{0.4:3,4}}
\def\Uabcd{\UPartition{0.4:1,2,3,4}{}}

%%%Partitions with same amount of upper and lower points

%%k=1, l=1

\def\Paa{\Partition{
\Pline (1,0) (1,1)
}}

\def\Pab{\Partition{
\Psingletons 0to0.3:1
\Psingletons 1to0.7:1
}}

%k=4

\def\Paaaa{\Partition{
\Pblock 0to0.3:1,2
\Pblock 1to0.7:1,2
\Pline (1.5,0.3) (1.5,0.7)
}}
\def\Paaab{\Partition{
\Pblock 1to0.7:1,2
\Psingletons 0to0.3:2
\Pline (1,0) (1,0.7)
}}
\def\Paaba{\Partition{
\Pblock 1to0.7:1,2
\Psingletons 0to0.3:1
\Pline (2,0) (2,0.7)
}}
\def\Pabaa{\Partition{
\Pblock 0to0.3:1,2
\Psingletons 1to0.7:1
\Pline (2,0.3) (2,1)
}}
\def\Pabbb{\Partition{
\Pblock 0to0.3:1,2
\Psingletons 1to0.7:2
\Pline (1,0.3) (1,1)
}}
\def\Paabb{\Partition{
\Pblock 0to0.3:1,2
\Pblock 1to0.7:1,2
}}
\def\Pabba{\Partition{
\Pline (1,0) (1,1)
\Pline (2,0) (2,1)
}}
\def\Pabab{\Partition{
\Pline (1,0) (2, 1)
\Pline (1,1) (2, 0)
}}
\def\Paabc{\Partition{
\Psingletons 0to0.3:1,2
\Pblock 1to0.7:1,2
}}
\def\Pabcc{\Partition{
\Pblock 0to0.3:1,2
\Psingletons 1to0.7:1,2
}}
\def\Pabac{\Partition{
\Psingletons 0to0.3:2
\Psingletons 1to0.7:1
\Pline (1,0) (2,1)
}}
\def\Pabcb{\Partition{
\Psingletons 0to0.3:1
\Psingletons 1to0.7:2
\Pline (2,0) (1,1)
}}
\def\Pabca{\Partition{
\Psingletons 0to0.3:1
\Psingletons 1to0.7:1
\Pline (2,0) (2,1)
}}
\def\Pabbc{\Partition{
\Psingletons 0to0.3:2
\Psingletons 1to0.7:2
\Pline (1,0) (1,1)
}}
\def\Pabcd{\Partition{
\Psingletons 0to0.3:1,2
\Psingletons 1to0.7:1,2
}}

%k=6

\def\Pabcabc{\Partition{
\Pline (1,0) (3, 1)
\Pline (2,0) (2, 1)
\Pline (3,0) (1, 1)
}}
\def\Pabcabd{\Partition{
\Pline (1,0) (3, 1)
\Pline (2,0) (2, 1)
\Psingletons 0to0.3:3
\Psingletons 1to0.7:1
}}
\def\Pabcadc{\Partition{
\Pline (1,0) (3, 1)
\Pline (3,0) (1, 1)
\Psingletons 0to0.2:2
\Psingletons 1to0.8:2
}}
\def\Pabcdbc{\Partition{
\Pline (2,0) (2, 1)
\Pline (3,0) (1, 1)
\Psingletons 0to0.3:1
\Psingletons 1to0.7:3
}}
\def\Pabcade{\Partition{
\Pline (1,0) (3, 1)
\Psingletons 0to0.2:2,3
\Psingletons 1to0.8:1,2
}}
\def\Pabcdec{\Partition{
\Pline (3,0) (1, 1)
\Psingletons 0to0.2:1,2
\Psingletons 1to0.8:2,3
}}
\def\Pabcdbe{\Partition{
\Pline (2,0) (2, 1)
\Psingletons 0to0.3:1,3
\Psingletons 1to0.7:1,3
}}
\def\Pabcdef{\Partition{
\Psingletons 0to0.3:1,2,3
\Psingletons 1to0.7:1,2,3
}}

\def\Paabaab{\Partition{
\Pblock 1to0.75:2,3
\Pblock 0to0.25:1,2
\Pline (1.5,0.25)(2.5,0.75)
\Pline (3,0)(1,1)
}}

%%%Some aliases

\def\singleton{{\mathord{\uparrow}}}
\def\upsingleton{{\mathord{\downarrow}}}
%\def\singleton{\LPartition{\Ls:1}{}}
%\def\upsingleton{\UPartition{\Us:1}{}}
\let\pairpart=\Laa
\let\uppairpart=\Uaa
\let\fourpart=\Laaaa
\let\idpart=\Paa
\let\disconnecterpart=\Pab
\let\connecterpart=\Paaaa
\let\crosspart=\Pabab
\let\positionerpart=\Pabcb
\let\halflibpart=\Pabcabc


%%%%%%%%%%%%%%%%%%%%%%%%%%%%%%%%%%%%%%%%%%%%%%%%%%%%
%%The macros

\def\Partition#1{%
\ifmmode
\mathchoice{
\,\PartitionA{0.25em}{0.7em}{1em}{#1}\,%
}{
\,\PartitionA{0.25em}{0.7em}{1em}{#1}\,%
}{
\,\PartitionA{0.175em}{0.55em}{0.7em}{#1}\,%
}{
\,\PartitionA{0.125em}{0.4em}{0.5em}{#1}\,%
}\else%
\thinspace\PartitionA{0.25em}{0.7em}{1em}{#1}\thinspace%
\fi%
}

\def\BigPartition#1{%
\ifmmode
\mathchoice{
\,\PartitionA{1em}{0.9em}{2.5em}{#1}\,%
}{
\,\PartitionA{0.75em}{0.7em}{2em}{#1}\,%
}{
\,\PartitionA{0.25em}{0.5em}{1em}{#1}\,%
}{
\,\PartitionA{0.20em}{0.4em}{0.8em}{#1}\,%
}
\else%
\thinspace\PartitionA{0.75em}{0.7em}{2em}{#1}\thinspace%
\fi%
}

\def\Pwidth{0.05}

\def\PartitionA#1#2#3#4{%
\pgfpicture
\pgfsetbaseline{#1}
\pgfsetxvec{\pgfpoint{#2}{0em}}
\pgfsetyvec{\pgfpoint{0em}{#3}}
\pgfsetlinewidth{\Pwidth em}
#4
\endpgfpicture
}

\def\fihere#1\fi{\fi#1}

\def\Pline(#1,#2)#3(#4,#5){
\pgfpathmoveto{\pgfpointxy{#1}{#2}}
\pgfpathlineto{\pgfpointxy{#4}{#5}}
\pgfusepath{stroke}
}

\def\Psingletons #1to#2:#3 {\PsingletonsA{#1}{#2}#3,,}
\def\PsingletonsA#1#2#3,{\ifx,#3,\else
\pgfpathmoveto{\pgfpointxy{#3}{#1}}
\pgfpathlineto{\pgfpointxy{#3}{#2}}
\pgfusepath{stroke}
\fihere\PsingletonsA{#1}{#2}\fi}

\def\Pblock #1to#2:#3 {\PblockA{#1}{#2}#3,,}
\def\PblockA#1#2#3,#4,{\ifx,#4,\else
\pgfpathmoveto{\pgfpointxy{#3}{#1}}
\pgfpathlineto{\pgfpointxy{#3}{#2}}
\pgfpathlineto{\pgfpointxy{#4}{#2}}
\pgfpathlineto{\pgfpointxy{#4}{#1}}
\pgfusepath{stroke}
\fihere\PblockA{#1}{#2}#4,\fi}

\def\Ptext(#1,#2)#3{%
\pgftext[at={\pgfpointxy{#1}{#2}}]{#3}
}

\def\Pcolw{0.1}
\def\Pb#1#2{
	\pgfpathcircle{\pgfpointxy{#1}{#2}}{\Pcolw em}
	\pgfsetfillcolor{black}
	\pgfusepath{stroke,fill}
}

\def\Pw#1#2{
	\pgfpathcircle{\pgfpointxy{#1}{#2}}{\Pcolw em}
	\pgfsetfillcolor{white}
	\pgfusepath{stroke,fill}
}

\def\Ls#1#2{
	\pgftransformshift{\pgfpointxy{#1}{#2}}
	\pgfsetarrowsend{to}
	\pgfpathmoveto{\pgfpointorigin}
	\pgfpathlineto{\pgfpointxy{0}{1}}
	\pgfusepath{stroke}
	\pgftransformreset
	\pgfsetarrowsend{}
}

\def\Us#1#2{
	\pgftransformshift{\pgfpointxy{#1}{#2}}
	\pgfsetarrowsend{to}
	\pgfpathmoveto{\pgfpointorigin}
	\pgfpathlineto{\pgfpoint{0}{-1}}
	\pgfusepath{stroke}
	\pgftransformreset
	\pgfsetarrowsend{}
}

\def\Ppoint #1 #2:#3 {\PpointA{#1}{#2}#3,,}
\def\PpointA#1#2#3,{\ifx,#3,\else #2{#3}{#1}\fihere\PpointA{#1}{#2}\fi}

\def\Lsingletons #1#2 {%
	\ifcat\noexpand#1\noexpand\ \Ppoint0 #1#2 \else\Psingletons 0to#1#2 \fi
}
\def\Usingletons #1#2 {%
	\ifcat\noexpand#1\noexpand\ \Ppoint1 #1#2 \else\Psingletons 1to#1#2 \fi
}


\def\LPartition#1#2{\Partition{
\LPartitionB#2;;
\LPartitionA#1;;
}}

\def\LPartitionA#1;{\ifx;#1;\else
\Lsingletons #1 
\fihere\LPartitionA\fi}

\def\LPartitionB#1;{\ifx;#1;\else
\Pblock 0to#1 
\fihere\LPartitionB\fi}

\def\UPartition#1#2{\Partition{
\UPartitionB#2;;
\UPartitionA#1;;
}}

\def\UPartitionA#1;{\ifx;#1;\else
\Usingletons #1 
\fihere\UPartitionA\fi}

\def\UPartitionB#1;{\ifx;#1;\else
\Pblock 1to#1 
\fihere\UPartitionB\fi}
